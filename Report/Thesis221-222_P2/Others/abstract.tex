\newpage
\vspace*{0.5cm}
\begin{center}
    \textbf{TÓM LƯỢC}
\end{center}

Trong những năm gần đây, sự phát triển của giao tiếp Machine-to-Machine (M2M) và Internet of Things (IoT) trong Công nghiệp đã mở đường cho việc phát triển các máy móc thông minh và tự điều khiển trong các doanh nghiệp sản xuất. Các máy tự động này có khả năng thu thập dữ liệu từ các thiết bị cảm biến khác nhau và các hệ thống máy được kết nối khác bằng cách sử dụng các giao thức truyền thông. Và để việc giao tiếp giữa các thiết bị được hiệu quả, nhanh chóng và đảm bảo về thời gian, cần phải chọn một giao thức kết nối phù hợp.

Hiện tại, những máy này có khả năng gửi và nhận dữ liệu đủ để đưa ra quyết định một cách tự chủ. Tuy nhiên, giao tiếp M2M ở cấp độ thấp như hiện tại thì chỉ dùng cho việc xử lý dữ liệu, và các hành động ra quyết định ở cấp độ cao hơn bị giới hạn trong một hệ thống máy riêng lẻ thay vì giao tiếp giữa các máy được kết nối với nhau. 

Từ nhu cầu cấp thiết về một loại giao thức truyền thông phù hợp cho môi trường Công nghiệp, đáp ứng sự đồng bộ cao, tốc độ nhanh và giải quyết được những nhược điểm đã nêu trên thì OPC-UA - một giao thức truyền tải dữ liệu chuẩn công nghiệp đã được đề xuất. Những hạn chế trong giao tiếp Machine-to-Machine sẽ được cải thiện bằng cách sử dụng giao thức OPC UA, giao thức liên quan đến việc đưa ra quyết định trong việc phản hồi dữ liệu nhận được và ra lệnh cho các máy được kết nối theo chiều dọc (máy cấp cao hơn) và chiều ngang (máy cùng cấp). Khi làm như vậy, các nhà máy sẽ trở nên thông minh hơn thông qua giao tiếp M2M chủ động hơn. Vì thế, Đồ án này sẽ tập trung vào việc nghiên cứu và hiện thực \textbf{Giao tiếp và điều khiển thiết bị dựa trên OPC-UA}.  Ngoài ra, để kiểm chứng được ưu thế của OPC-UA thì Đồ án này sẽ thực hiện so sánh hiệu suất về thời gian gửi và nhận tin nhắn của hai giao thức nổi tiếng trong lĩnh vực IoT và IoT công nghiệp (IIoT) - MQTT và OPC-UA. Từ đó, tận dụng những ưu điểm của OPC-UA, một ứng dụng Digital Twin cũng sẽ được xây dựng để phục vụ trong môi trường Công nghiệp.  \\
% Các giá trị thu được cho thấy lợi thế của việc sử dụng từng giao thức trong một số tình huống, nhằm khai thác khả năng trao đổi dữ liệu tốt nhất giữa các thiết bị.\\

\underline{Các từ khóa :} OPC-UA, MQTT, M2M, IoT, Công nghiệp 4.0, Digital Twin, Cánh tay Robot, VR application.